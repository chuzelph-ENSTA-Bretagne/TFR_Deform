
\chapter{Spectrogram}

The spectrogram is the most common tools used in order to do a time frequency representation. This representation use the short-time-Fourier transformation in order to apply an fast-Fourier transformation on a slippery window of the signal. In the end, we get an image which represent the frequency which compose the signal at an instant t given.

Let consider the signal chirp below which correspond to a signal which its frequency increases from 0 to 300Hz.

\begin{figure}[H]
\centering
    \includegraphics[scale=0.6,angle=0]{ChirpTF.png}
    \caption{Time and frequency representation of a chirp signal.}
    \label{fig:ChirpTF}
\end{figure}

The figure below shows the results we can get:

\begin{figure}[H]
\centering
    \includegraphics[scale=0.6,angle=0]{Chirp_STFT.png}
    \caption{STFT/spectrogram of a chirp signal.}
    \label{fig:Chirp_STFT}
\end{figure}

This figure shows clearly that the frequency of the signal increass through time but the result is a little blur and isn't very accurate.

\chapter{Wigner-Ville}

The Wigner-Ville distribution allows us to get a far more accurate time frequency representation. Nevertheless, this distribution create some interferences if the signal is a combination of different frequency laws.

The Wigner-Ville Distribution (WVD) of a signal y(t), denoted by $W_z(t,f)$, is defined as :

\begin{equation}
W_z(t,f) = \int_{n=-\infty}^{\infty} z ( t + \tau / 2 ) z^* (t - \tau / 2) e^{-j 2 \pi f \tau d \tau}
\end{equation}

The following figure shows the result of the Wigner-Ville representation on the previous signal:

\begin{figure}[H]
\centering
    \includegraphics[scale=0.5,angle=0]{Chirp_WV.png}
    \caption{Wigner-Ville of a chirp signal.}
    \label{fig:Chirp_WV}
\end{figure}

Nevertheless, if we consider the following signal which contains two sinusoids, with different frequency and with one which last for only a portion of the signal. We get the result below:

\begin{figure}[H]
\centering
    \includegraphics[scale=0.5,angle=0]{SignalSimple.png}
    \caption{Time and frequency representation of the signal with two frequency laws.}
    \label{fig:SignalSimple}
\end{figure}

\begin{figure}[H]
\centering
    \includegraphics[scale=0.5,angle=0]{figure_6STFT.png}
    \caption{Spectrogram of the signal.}
    \label{fig:SignalSimple_Spectrogram}
\end{figure}

\begin{figure}[H]
\centering
    \includegraphics[scale=0.5,angle=0]{SignalSimple_WV.png}
    \caption{Wigner-Ville of the signal.}
    \label{fig:SignalSimple_WV}
\end{figure}

As you can see, there are an interference which imply to analyse this figure if we want to extract relevant information.

\textbf{NB :} The differences in the frequency scale is due to a problem in the implementation I realized. I will correct this as soon as I can.

\chapter{Pseudo Smooth Wigner-Ville}

The  Pseudo-Wigner-Ville Distribution is defined as:

\begin{equation}
W_z(t,f) = \int_{n=-\infty}^{\infty} h ( \tau ) z ( t + \tau / 2 ) z^* (t - \tau / 2) e^{-j 2 \pi f \tau d \tau}
\end{equation}

where $h$ is a regular window. This windowing is equivalent to a frequency smoothing of the WVD so It leads to the attenuation of the interference terms but it will damage the signal representation.

The result for the previous signal is below:

\begin{figure}[H]
\centering
    \includegraphics[scale=0.5,angle=0]{SignalSimple_PSWV.png}
    \caption{Pseudo Smooth Wigner-Ville of the signal.}
    \label{fig:SignalSimple_PSWV}
\end{figure}



\paragraph{Conclusion}
\bigskip
All those representation present some interests or disadvantages. Nevertheless, if we want to get as much information as we can, we will need to switch between a representation to another according to the kind of signal we have.

