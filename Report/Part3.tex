


\chapter{Seam carving algorithm}

For this part, most of the tools come from those references \cite{Doopler1} \cite{Doopler2} \cite{Doopler3}.
\textbf{For now, I'm still working on this part so I will complete this part as soon as I get result.}

At the beginning, the seam carving algorithm is used in order to reduce the size of an image and keep some proportion of the image. Nevertheless, it can be used in this project in order to find a approximation of the frequency law which compose a signal.
The main purpose of those article is to present an algorithm which allow us to find the speed of an object thanks to the Doppler effect but it can be use here in order to find the frequency laws which compose the signal. With this, we can find which are the component of a signal. In the end, we have a way to find correlation between the electromagnetic signature and the kind of from changes of an object.

The seam carving algorithm can be express by the following line:

\begin{algorithm}
  \caption{SEAM CARVING algorithm }
  
  \textbf{Inputs}% Inputs section
  \begin{algorithmic}[1]
    \STATE Matrix TFR
    \STATE Vector $T$ time
    \STATE Vector $F$ frequency
  \end{algorithmic}
  \bigskip

  \textbf{Output}% Output section
  \begin{algorithmic}[1]
    \STATE Vector $Value f(t) of the first frequency law found$
  \end{algorithmic}
  \bigskip
  
  \textbf{Initialization}% Initialization section
  \begin{algorithmic}[1]
   	\STATE $init\gets 0$
   	\STATE $i\gets 0$
	\STATE $k\gets zeros(len(t),1)$
  \end{algorithmic}
  
  
  \textbf{Algorithm}% Inputs section
  \begin{algorithmic}[1]
	\WHILE{i<len of T}

	 	
	 	\IF{i==0 or init == 0}
     	  	\STATE $peakind =findPeaksCwt($TFR$[:,i])$
     	
	 	
  	 		\IF{$len(peakind)!=0$}
     	  		\STATE $tmp[peakind[0],i] = 1$
				\STATE $k[i] = peakind[0]$
				\STATE $init=1$
				
	 	
			\ELSE{}
				\IF{max(TFR[:,i])!=0)}
     	  			\STATE $a,ind = max($TFR$[:,i]) $
					\STATE $k[i]=ind$
					\STATE $init=1$
				\ENDIF
			\ENDIF
				
		\ELSE
			\STATE $Indice = 0$
			\STATE $Vartmp = 0$
			\FOR{$j\in [-1,0,1]$}
  	 			\IF{$Vartmp<$TFR$[k[i-1]+j,i]$}
					\STATE $Vartmp = $TFR$[k[i-1]+j,i]$
					\STATE $Indice = j$
				\ENDIF
				\STATE $k[i] = k[i-1] + Indice$
				\STATE $tmp[k[i],i] = 1$
			\ENDFOR
  	 	
	 	\ENDIF
	 	\STATE $i++$
	\ENDWHILE 
	
	
  return $k$


  \end{algorithmic}
\end{algorithm}

\chapter{Application on our project}

We will now apply this algorithm to the time frequency representation coming from the part 2 and from the signal generate in the part 1.
\textbf{For now, I'm still working on this part so I will complete this part as soon as I get result.}